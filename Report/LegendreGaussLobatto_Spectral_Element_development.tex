%%%%%%%%%%%%%%%%%%%%%%%%%%%%%%%%%%%%%%%%%%%%%%%%%%%%%%%%%%%%%%%%%%%%%
\documentclass[12pt]{article}% insert '[draft]' option to show overfull boxes
\usepackage{amsmath}
\usepackage{graphicx,subfigure}
\usepackage[footnotesize]{caption}
\usepackage{wrapfig}

\title{Development of a Spectral-Element code}

\author{Tyler Arsenault\\ tyler.j.arsenault@gmail.com}
\date{}
%

\begin{document}

\maketitle

Using Legendre polynomials on a Gauss-Lobatto quadrature.

\[P_n(x)=\frac{2n-1}{n}xP_{n-1}(x)-(n-1)P_{n-2}(x)\]
%
\[P'_n(x)=(2n-1)P_{n-1}(x)+P'_{n-2}(x)\]
where $P'_0(x)=0$ and $P'_1(x)=1$


\subsubsection*{Define a basis function}
We need to define a basis function.  We will define it as the Lagrange interpolation polynomial (Equation 5.46 in Hestaven)
\[l_j(x)=\frac{-1}{N(N+1)}\frac{1-x^2}{x-x_j}\frac{P'_N(x)}{P_N(x_j)}
.\]

This will cause a problem, $\lim {x\to x_j}=\infty$
Have to use L'hopital.

The numerator:
\[-\frac{d}{dx} (1-x^2) P'_N(x)=-P''_N(x)+(x^2P''_N(x)+2xP'_N(x))\]
\[-P''_N(x)(1-x^2)+2xP'_N(x)\]

The denominator:
\[\frac{d}{dx}N(N+1)(x-x_j)P_N(x_j)=N(N+1)P_N(x_j)\]

Now we have:
\[l_j=\frac{-P''(x)(1-x^2)+2xP'(x)}{N(N+1)P_N(x_j)}\]

This will become our basis,
\[\phi_n=\frac{-P''_N(x)(1-x^2)+2xP'_N(x)}{N(N+1)P_N(x_n)}\]
I'm confused here. Is $N$ only the $N^{th}$ order polynomial or is this a polynomial sum which should look like,
\[\phi_n=\sum_{i=0}^N \frac{-P''_i(x)(1-x^2)+2xP'_i(x)}{N(N+1)P_i(x_n)}\]

\subsection*{Weak formulation}

Refering to p.120 in Hestaven.

%The basis function we will choose is:
%\[\phi_n(x_i)=P_{n+1}(x_i)-P_{n-1}(x_i)\]

This will be used in our approximation with coefficients $a_n(t_i)$ as,

\[u_N(x,t)=\sum_{n=1}^{N-1}a_n(t)\,\phi_n(x_i)\]

\[\frac{\partial}{\partial t} u_N(x,t)=\sum_{n=1}^{N-1}a_n'(t)\,\phi_n(x_i)\]

\[\frac{\partial ^2}{\partial x ^2} u_N(x,t)=\sum_{n=1}^{N-1}a_n(t)\,\phi_n''(x_i).\]



\[ \frac{\partial}{\partial t}  u_N+\frac{\partial ^2}{\partial x ^2}u_N=f(x)\]

we take use the weak, so we take the inner product with respect to a test function $\phi_m(x)$,

\[\int_{-1}^{1} \left(a_n'(t)\,\phi_n(x_i) + a_n(t)\,\phi_n''(x_i)\right) \phi_m(x_i)\,dx=\int_{-1}^{1} f(x_i)\phi_m(x)\,dx\]

\[\int_{-1}^{1} a_n'(t)\,\phi_n(x_i)\,\phi_m(x_i)\,dx + \int_{-1}^{1}a_n(t)\,\phi_n''(x_i) \phi_m(x_i)\,dx=\int_{-1}^{1} f(x_i)\phi_m(x)\,dx.\]

The mass term $\int_{-1}^{1} a_n'(t)\, \phi_n(x_i)\,
\phi_m(x_i)\,dx$ will be given as  
\[M_{m,n}=\begin{cases} w(x_i)& m=n \\ 0 & \neq n\end{cases}\]
where $\int_{-1}^{1} a_n'(t)\, \phi_n(x_i)\,
\phi_m(x_i)\,dx=a'_n(t)M_{m,n}$.


The symmetric stiffness term $\int_{-1}^{1}a_n(t)\,\phi_n''(x_i) \phi_m(x_i)\,dx$ will be integrated by parts to get,

\[\phi_n(x_i) \phi_m'(x_i)\Big|_{-1}^1-\int_{-1}^{1}a_n(t)\,\phi_n'(x_i) \phi'_m(x_i)\,dx\]

The stiffness matrix term is

\[K_{m,n}=\begin{cases}
-\sum_{i=1}^N\phi'_n(x_i)\phi'_m(x_i)w_i &  0<n<N
\\
????? &  n=0\\
????? &  n=N
\end{cases}\]


\end{document}
