%%%%%%%%%%%%%%%%%%%%%%%%%%%%%%%%%%%%%%%%%%%%%%%%%%%%%%%%%%%%%%%%%%%%%
\documentclass[12pt]{article}% insert '[draft]' option to show overfull boxes
\usepackage{amsmath}
\usepackage{graphicx,subfigure}
\usepackage[footnotesize]{caption}
\usepackage{wrapfig}
\usepackage{color}


\title{Development of a Spectral-Element code}

\author{Tyler Arsenault\\ tyler.j.arsenault@gmail.com}
\date{}
%

\begin{document}

\maketitle

Using Legendre polynomials on a Gauss-Lobatto quadrature.

\begin{equation}
P_n(x)=\frac{2n-1}{n}xP_{n-1}(x)-(n-1)P_{n-2}(x)
\end{equation}
%
\begin{equation}P'_n(x)=(2n-1)P_{n-1}(x)+P'_{n-2}(x)
\end{equation}
where $P'_0(x)=0$ and $P'_1(x)=1$


\subsubsection*{Define a basis function}
We need to define a basis function.  We will define it as the Lagrange interpolation polynomial (Equation 5.46 in Hesthaven)
\begin{equation}
\phi_j(x)=\frac{-1}{n(n+1)}\frac{1-x^2}{x-x_j}\frac{P'_n(x)}{P_n(x_j)}
.\end{equation}


\section{Galerkin formulation}
Using the equation for the 1D Helmoltz,
\begin{equation}
u_t+u_{xx}=f(x)
\end{equation}

Take the inner product with respect to a test function $\phi_m(x)$,
\begin{equation}
\int_{\Omega} \left(u_t + u_{xx}\right) \phi_m(x)\,d\Omega=\int_{\Omega} f(x)\phi_m(x)\,d\Omega
\end{equation}

\begin{equation}\int_{\Omega} u_t\phi_m(x)\,d\Omega +\int_{\Omega}u_{xx} \phi_m(x)\,d\Omega=\int_{\Omega} f(x)\phi_m(x)\,d\Omega.
\end{equation}

We are going to approximate the solution with,
\begin{equation}
u_N(x,t)=\sum_{n=0}^Na_n(t)\phi_n(x)
\end{equation}
\section{Unsteady Term}
The unsteady term becomes 
\begin{eqnarray}
\frac{1}{\gamma_m}\sum_{n=0}^N \int_{-1}^{1} a_n'\phi_n\,\phi_m w(x)\,dx \\
\frac{1}{\gamma_m}\sum_{n=0}^N a_n'\int_{-1}^{1}\phi_n\,\phi_m w(x)\,dx\\
\end{eqnarray}
The integral of the product of the basis functions will be zero except when $m=n$, where it will be $w(x)$. Giving the unsteady term as 
\begin{equation}M_{m,n}=\begin{cases} w(x)& m=n \\ 0 & m\neq n\end{cases}
\end{equation}
We have the unsteady term as
\begin{equation}
\frac{1}{\gamma_m}M_{m,n}a'(t)
\end{equation}

\section{The spatial Term}
The spacial derivative term $\int_\Omega u_{xx} \phi_m(x)\,d\Omega$ will be integrated by parts to get to show how this works between two elements we will use the domain $-3\leq x \leq 1$ where a unit element is $-1\leq x \leq 1$.
\begin{equation}
\int_{-3}^1 u_{xx} \phi_m\,dx
\end{equation}

\begin{equation}
\int_{-3}^{-1} u_n(t)\,\phi_n''(x) \phi_m(x)\,dx+\int_{-1}^1 u_n(t)\,\phi_n''(x) \phi_m(x)\,dx
\end{equation}
integrating by parts,
\begin{align}
u^{(i)}_n(t)\,\phi_n'(x)& \phi_m(x)\Big|_{-3}^{-1}-\int_{-3}^{-1} u^{(i)}_n(t)\,\phi_n'(x) \phi'_m(x)\,dx...\\
&+u_n^{(i+1)}(t)\,\phi_n'(x) \phi_m(x)\Big|_{-1}^{1}-\int_{-1}^{1} u^{(i+1)}_n(t)\,\phi_n'(x) \phi'_m(x)\,dx.\nonumber
\end{align}
Now evaluate with the basis being $1$ at $x=-1$ and zero otherwise.
\begin{align}
u_n^{(i)}(t)\,\phi_n'(-1)& \phi_m(-1)-\int_{-3}^{-1} u_n^{(i)}(t)\,\phi_n'(x) \phi'_m(x)\,dx...\\
&-u^{(i+1)}_n(t)\,\phi_n'(-1) \phi_m(-1)-\int_{-1}^{1} u_n^{(i+1)}(t)\,\phi_n'(x) \phi'_m(x)\,dx\nonumber
\end{align}
\begin{equation}
-\int_{-3}^{-1} u^{(i)}_n(t)\,\phi_n'(x) \phi'_m(x)\,dx-\int_{-1}^{1} u^{(i+1)}_n(t)\,\phi_n'(x) \phi'_m(x)\,dx
\end{equation}
Its discretization time!!! and addition of the element counter $i$, where $i=0,1,2,...,K$. Replace $j$ with $n$ on the basis. blah.
\begin{equation}
u^{(i)}_n=\sum_{j=0}^N u^{(i)}_j\phi_j(x)
\end{equation}
\begin{equation}
\sum_{j=0}^N u^{(i)}_j\left[-\int_{-3}^{-1} \phi_n'(x) \phi'_m(x)\,dx\right]+\sum_{j=0}^N u^{(i+1)}_j\left[-\int_{-1}^{1} \phi_n'(x) \phi'_m(x)\,dx\right].
\end{equation}
We see the term $\int_{-1}^{1} \phi_n'(x) \phi'_m(x)\,dx$ is repeating.
The elemental stiffness matrix will take the form
\begin{equation}
K_{m,n}=-\int_{-1}^{1} \phi_n'(x) \phi'_m(x)\,dx.
\end{equation}

From Hesthaven p. 95 the differentiation matrix takes the form
\begin{equation}
D_{ij}=\begin{cases}
-\frac{N(N+1)}{4} &i=j=0\\
0& i=j\in [1,...,N-1]\\
\frac{P_N(x_i)}{P_N(x_j)}\frac{1}{x_i-x_j} & i\neq j\\
\frac{N(N+1)}{4} & i=j=N
\end{cases}
\end{equation}
This means that we can approximate 
\begin{equation}
\frac{d}{dx}\,u = D_1 \,u
\end{equation}
\textcolor{red}{\textbf{this also means that correct?}}
\begin{equation}
\frac{d^2}{dx^2}\,u = D_2 \,u=D_1\,D_1 \,u
\end{equation}
This means the stiffness matrix is just $D_1\,D_1$??







\end{document}
